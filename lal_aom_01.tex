\NeedsTeXFormat{LaTeX2e}
%\documentclass[manuscript]{aomart}
%\documentclass[screen]{aomart}
\documentclass{aomart}
\usepackage[english]{babel}
\usepackage{array}
\usepackage{colortbl}
\usepackage{longtable}
\usepackage{color}
\usepackage{hhline}
\usepackage{adjustbox}
\usepackage{amsmath}
\usepackage{amssymb}
\usepackage{fancyhdr}
\usepackage{float}
\usepackage[T1]{fontenc}
\usepackage{graphicx}
\usepackage{hhline}
\usepackage[utf8]{inputenc}
\usepackage{mathtools}
\usepackage{multicol}
\usepackage{multirow}
\usepackage{txfonts}
\usepackage{wasysym}
\usepackage{appendix}
%\usepackage[hidelinks]{hyperref}
\usepackage[svgnames]{xcolor}
\setcounter{tocdepth}{5}

\title[A Novel Sieve Approach to Twin Primes]{A Novel Sieve Approach to Twin Primes}
\address{S. Ao\\
University of Regina, SK }
\email{ao.suqin@gmail.com} 
\author[S. Ao and L. Lu]{Suqin Ao and Lucy Lu}
\fulladdress{3737 Wascana Pkwy, Regina, SK S4S 0A2, Canada}
\urladdr{https://www.uvic.ca}
\givenname{Suqin}
\surname{Ao}
\copyrightyear{2023}
\copyrightnote{\textcopyright~2023 Suqin Ao}
\thanks{}

%\keyword{twin-primes prime-pairs}
%\keyword{twin-primes}
\subject{primary}{mathematics}{twin primes}
\subject{secondary}{number theory}{prime pairs}

%\received{\formatdate{2023-12-24}}
%\revised{\formatdate{2023-04-12}}
%\accepted{\formatdate{2023-10-02}}
%\published{\formatdate{2023-04-02}}
%\publishedonline{\formatdate{2023-11-12}}

\volumenumber{000}
\issuenumber{0}
\publicationyear{2023}
\papernumber{01}
\startpage{1}
\endpage{}

\proposed{S. Ao}
\seconded{L. Lu}
\corresponding{L. Lu}
\version{2.1}

\doinumber{00.1111/S0000-1111-22-3333-4}
\mrnumber{MR123456}
\zblnumber{0000.1111}

%\oldsubsections

\begin{document}
\begin{abstract}
  The distribution and properties of prime pairs, especially twin primes, have been central questions in number theory. Utilizing a unique sieve method, this paper introduces a fresh perspective on the study of these prime pairs. Our methodology not only sheds light on their distribution and inherent properties but also provides a proof that for every natural number \( k \), an infinite number of prime pairs exist with a separation of \( 2k \) units.  

\end{abstract}

\maketitle
\tableofcontents

\section{Introduction}

A \textit{prime number} (or a \textit{prime}) is a natural number greater than 1, that has no positive divisors other than 1 and itself.  
\vspace{1\baselineskip}

We introduce the following notations:

\begin{itemize}
    \item \( N \): The set of all natural numbers.
    \item \( P[X] \): The set of all prime numbers in the set of \( X \).
    \item \( P \): Equivalent to \( P[N] \), representing the set of all prime numbers.
    \item \( \{ u_{i} \} \): A sequence with a general term \( u_{i} \).
    \item \( \ln x \): The natural logarithm of \( x \).
    \item \( \pi (x) \): The number of primes less than or equal to \( x \).
    \item \( P(x) \): The set of all primes less than or equal to \( x \), represented as \( \{ 2,3,5,\ldots p \} \)
    \item \( \text{gcd}(a,b) \): The greatest common divisor of \( a \) and \( b \).

\end{itemize}

\vspace{1\baselineskip}

We say that \( a \) and \( b \) are \textit{relatively prime} if \( \text{gcd}(a,b) = 1 \) \cite{2}.


\begin{flushleft}
The  {\textit{Sieve of Eratosthenes} }([1]) has long been used in the study of prime numbers. 
\end{flushleft}


For a large positive number \( n\), let \( S = \{ m,  1<m\leq n\}\) { } be the set of natural numbers great than 1 and less than or equal to \( n\) {.  Then we can use the sieve of Eratosthenes to find all prime numbers in the set }S such that:

\begin{equation}
    (1)                              P\left[S\right] =  S- \bigcup_{p\in P\left(\sqrt{n}\right)}^{} S_{p}  ,
\end{equation}

\begin{equation}
\text{where } S_{p} = \left\{ mp, m \in \mathbb{N} \text{ and } p \leq m \leq \frac{n}{p} \right\} , \text{ for all } p \in P(\sqrt{n}).
\end{equation}


\vspace{1\baselineskip}

We introduce \textit{Euler's} \( \phi \)-function, \( \phi (n) \), the number of natural numbers less than or equal to \( n \) that are relatively prime to \( n \) \cite{2}, \cite{1}.

\vspace{1\baselineskip}

For any large positive number \( x \), let 

\begin{equation}
N = \prod_{p\in P\left(\sqrt{x}\right)}^{}p  ,
\end{equation}

\vspace{1\baselineskip}

The product of all primes less than or equal to \(\sqrt{x}\) is given as:
\begin{equation}
N = \prod_{p \leq \sqrt{x}} p.
\end{equation}
\vspace{1\baselineskip}

Let
\begin{equation}
X(z) = \left\{ m \mid 1 \leq m \leq z \text{ and } \gcd(m,N) = 1 \right\}, \text{ for } 1 < z \leq N,
\end{equation}
\vspace{1\baselineskip}

be the set of natural numbers less than or equal to \(z\) that are relatively prime to \(N\).

\vspace{1\baselineskip}

From the definition of the \( \phi \)-function, the size of \( X(N) \) is given by:

\begin{equation}
\phi\left(N\right) = N\prod_{p\in P\left(\sqrt{x}\right)}^{}(1-\frac{1}{p} )  .
\end{equation}
\vspace{1\baselineskip}

 Notice that all numbers in the set \( X\left(x\right)\) are primes and \( X\left(x\right) = P\left(x\right)- P\left(\sqrt{x}\right)+\{ 1\}\)\textit{,}

 therefore, the { size of }\( X\left(x\right)\) is {:}

\begin{equation}
\pi\left(x\right)-\pi\left(\sqrt{x}\right)+1 = C_{1}\frac{x}{N} \phi\left(N\right) = C_{1}x\prod_{p\in P\left(\sqrt{x}\right)}^{}(1-\frac{1}{p} ) ,
\end{equation}
\vspace{1\baselineskip}

for some constant \( C_{1}\). 
\vspace{1\baselineskip}


For Example:  take  \( x = 100\),  then  \(\sqrt{x} = 10,\) \( P\left(\sqrt{x}\right) =\left\{ 2,3,5,7\right\}\), \( N = 2\cdot 3\cdot 5\cdot 7 = 210\),

\begin{equation}
X\left(210\right) =\left\{  1, 11, 13, \ldots , 97,\ldots ,209\right\}  = P\left(210\right)- P\left(10\right)+\left\{ 1, 121, 143, 169, 187, 209\right\} 
\end{equation}


\begin{equation}
\phi\left(210\right) = 210\left(1-\frac{1}{2}\right)\left(1-\frac{1}{3}\right)\left(1-\frac{1}{5}\right)\left(1-\frac{1}{7}\right) = 48 ,  
\end{equation}


\begin{equation}
X\left(100\right) =\left\{  1, 11, 13, 17, \ldots , 97\right\}  = P\left(100\right)- P\left(10\right)+\left\{ 1\right\} ,
\end{equation}
\vspace{1\baselineskip}

the\  { size of }\( X\left(100\right)\) is {:} \( \pi\left(100\right)-\pi\left(10\right)+1 = 25-4+1 = 22,\) compare this with

\begin{equation}
C_{1}\frac{100}{210} \phi\left(210\right) = C_{1}\frac{100}{210} 48\approx 22.86C_{1}  
\end{equation}
\vspace{1\baselineskip}

we expect that \( 0<C_{1}<1\).
\vspace{1\baselineskip}

So, if we use the  {Sieve to S to get }\( P\left[S\right]\) { }as (1), then we have the following approximations of \( \pi\left(n\right)\), the size of \( P\left[S\right]\):

\begin{equation}
C_{1} n\prod_{p\leq\sqrt{n}}^{}\left(1-\frac{1}{p}\right) = C_{1} n\left(1-\frac{1}{2}\right)\left(1-\frac{1}{3}\right)\ldots \left(1-\frac{1}{q}\right) ,     2<3<\ldots <q\leq\sqrt{n} .  
\end{equation}

\vspace{1\baselineskip}

\ \ \ \ In fact, from ([1]), we know that \( C_{1}\) is about \(\frac{e^{\gamma }}{2}\approx 0.890536209\ldots\),  where $\gamma$ is Euler’s constant,

\begin{equation}
  and          \prod_{p\in P\left(\sqrt{x}\right)}^{}(1-\frac{1}{p} ) \sim  \frac{2e^{-\gamma }}{\ln \ln  x }  ,
\end{equation}

\vspace{1\baselineskip}

\ \ \ \ so, we have

\begin{equation}
\pi\left(x\right) \sim  \frac{e^{\gamma }}{2}  x\prod_{p\in P\left(\sqrt{x}\right)}^{}\left(1-\frac{1}{p}\right) \sim  \frac{x}{\ln \ln  x }   ,
\end{equation}
\vspace{1\baselineskip}

 or 
\begin{equation}
\pi(x) \sim Li(x) = \int_{2}^{x} \frac{dt}{\ln t},
\end{equation}

\section{Prime Pairs}



Now we study the prime pairs.
\vspace{1\baselineskip}
For \( k\in N\) both \( p \ and \ p+2k\) are primes.\textbf{\ \ }We know that twin primes are the special case of \( k = 1\).
\vspace{1\baselineskip}

 To find all twin primes \( (p,  p+2)\) such that \( p\leq n\), we only need to find the first prime \textit{p} in the pair by using  {the Sieves to the set }\( S = \{ m,  1<m\leq n\}\) as showing in the following  {set}:

\begin{equation}
\left(2\right)                  P_{1}P\left[S\right] =  S- \bigcup_{p\in P\left(\sqrt{n}\right)}^{} S_{p} - \bigcup_{p\in P\left(\sqrt{n+2}\right)}^{}S_{p}\left(2\right) =  P\left[S\right]-\bigcup_{p\in P\left(\sqrt{n+2}\right)}^{}S_{p}\left(2\right) ,
\end{equation}

\begin{equation}
\text{where } S_{p} = \left\{ mp \, | \, m \in \mathbb{N} \text{ and } p \leq m \leq \frac{n}{p} \right\} \text{ for all } p \in P(\sqrt{n}),
\end{equation}

\begin{equation}
\text{and } S_{p}(2) = \left\{ mp-2 \, | \, m \in \mathbb{N} \text{ and } p \leq m \leq \frac{n+2}{p} \right\} \text{ for all } p \in P(\sqrt{n+2}),
\end{equation}

\vspace{1\baselineskip}

For Example:
\begin{center}
\begin{tabular}{ll}
when \( n = 100\), & \\
\( S \)              = \{ 2, 3, \ldots , 99, 100\},                & \\
\( S_{2} \)          = \{ 4, 6, 8, \ldots , 98, 100\},            & \( S_{2}(2) \)       = \{ 2, 4, 6, \ldots , 96, 98, 100\}, \\
\( S_{3} \)          = \{ 9, 12, 15, \ldots , 96, 99\},           & \( S_{3}(2) \)       = \{ 7, 10, 13, \ldots , 94, 97, 100\}, \\
\( S_{5} \)          = \{ 25, 30, 35, \ldots , 95, 100\},         & \( S_{5}(2) \)       = \{ 23, 28, 33, \ldots , 93, 98\}, \\
\( S_{7} \)          = \{ 49, 56, 63, \ldots , 91, 98\},          & \( S_{7}(2) \)       = \{ 47, 54, 61, \ldots , 89, 96\}. \\
\end{tabular}
\end{center}
\vspace{1\baselineskip}
From
\begin{equation}
  P_{1}P\left[S\right] = S-\left\{ S_{2}\cup  S_{2}\left(2\right)\cup  S_{3}\cup  S_{3}\left(2\right)\cup  S_{5}\cup  S_{5}\left(2\right)\cup  S_{7}\cup  S_{7}\left(2\right)\right\}  = \{ 3, 5, 11, 17, 29, 41, 59, 71\} ,
\end{equation}


we find out all 8 twin primes less than 100:

\begin{equation}
\left\{\left(3, 5\right),\left(5, 7\right),\left(11, 13\right),\left(17, 19\right),\left(29, 31\right),\left(41, 43\right),\left(59, 61\right),\left(71, 73\right)\right\} .
\end{equation}

\vspace{1\baselineskip}

See the following Table 1. for some examples:


\begin{table}[H]
\centering
\begin{adjustbox}{max width=\textwidth}
\begin{tabular}{|c|c|c!{\vrule width 1.5pt}c|c|c!{\vrule width 2pt}c|c|c!{\vrule width 1.5pt}c|c|c!{\vrule width 1.5pt}c|c|c|}
\hline
s & n & n+2 & s & n & n+2 & s & n & n+2 & s & n & n+2 & s & n & n+2 \\ \hline

  &   & 3 & \cellcolor{red!30} 21 & \cellcolor{red!30} 21 & 23 & 41 & 41 & 43 & \cellcolor{red!30} 61 & 61 & \cellcolor{red!30} 63 & \cellcolor{red!30} 81 & \cellcolor{red!30} 81 & 83 \\ \hline
\cellcolor{red!10} 2 & 2 & \cellcolor{red!10} 4 & \cellcolor{red!10} 22 & \cellcolor{red!10} 22 & \cellcolor{red!10} 24 & \cellcolor{red!10} 42 & \cellcolor{red!10} 42 & \cellcolor{red!10} 44 & \cellcolor{red!10} 62 & \cellcolor{red!10} 62 & \cellcolor{red!10} 64 & \cellcolor{red!10} 82 & \cellcolor{red!10} 82 & \cellcolor{red!10} 84 \\ \hline


3 & 3 & 5 & \cellcolor{red!50} 23 & 23 & \cellcolor{red!50} 25 & \cellcolor{red!30} 43 & 43 & \cellcolor{red!30} 45 & \cellcolor{red!30} 63 & \cellcolor{red!30} 63 & \cellcolor{red!50} 65 & \cellcolor{red!50} 83 & 83 & \cellcolor{red!50} 85 \\ \hline
\cellcolor{red!10} 4 & \cellcolor{red!10} 4 & \cellcolor{red!10} 6 & \cellcolor{red!10} 24 & \cellcolor{red!10} 24 & \cellcolor{red!10} 26 & \cellcolor{red!10} 44 & \cellcolor{red!10} 44 & \cellcolor{red!10} 46 & \cellcolor{red!10} 64 & \cellcolor{red!10} 64 & \cellcolor{red!10} 66 & \cellcolor{red!10} 84 & \cellcolor{red!10} 84 & \cellcolor{red!10} 86 \\ \hline
5 & 5 & 7 & \cellcolor{red!30} 25 & \cellcolor{red!50} 25 & \cellcolor{red!30} 27 & \cellcolor{red!30} 45 & \cellcolor{red!30} 45 & 47 & \cellcolor{red!50} 65 & \cellcolor{red!50} 65 & 67 & \cellcolor{red!30} 85 & \cellcolor{red!50} 85 & \cellcolor{red!30} 87 \\ \hline
\cellcolor{red!10} 6 & \cellcolor{red!10} 6 & \cellcolor{red!10} 8 & \cellcolor{red!10} 26 & \cellcolor{red!10} 26 & \cellcolor{red!10} 28 & \cellcolor{red!10} 46 & \cellcolor{red!10} 46 & \cellcolor{red!10} 48 & \cellcolor{red!10} 66 & \cellcolor{red!10} 66 & \cellcolor{red!10} 68 & \cellcolor{red!10} 86 & \cellcolor{red!10} 86 & \cellcolor{red!10} 88 \\ \hline
\cellcolor{red!30} 7 & 7 & \cellcolor{red!30} 9 & \cellcolor{red!30} 27 & \cellcolor{red!30} 27 & 29 & \cellcolor{red!70} 47 & 47 & \cellcolor{red!70} 49 & \cellcolor{red!30} 67 & 67 & \cellcolor{red!30} 69 & \cellcolor{red!30} 87 & \cellcolor{red!30} 87 & 89 \\ \hline
\cellcolor{red!10} 8 & \cellcolor{red!10} 8 & \cellcolor{red!10} 10 & \cellcolor{red!10} 28 & \cellcolor{red!10} 28 & \cellcolor{red!10} 30 & \cellcolor{red!10} 48 & \cellcolor{red!10} 48 & \cellcolor{red!10} 50 & \cellcolor{red!10} 68 & \cellcolor{red!10} 68 & \cellcolor{red!10} 70 & \cellcolor{red!10} 88 & \cellcolor{red!10} 88 & \cellcolor{red!10} 90 \\ \hline
\cellcolor{red!30} 9 & \cellcolor{red!30} 9 & 11 & 29 & 29 & 31 & \cellcolor{red!30} 49 & \cellcolor{red!70} 49 & \cellcolor{red!30} 51 & \cellcolor{red!30} 69 & \cellcolor{red!30} 69 & 71 & \cellcolor{red!70} 89 & 89 & \cellcolor{red!70} 91 \\ \hline
\cellcolor{red!10} 10 & \cellcolor{red!10} 10 & \cellcolor{red!10} 12 & \cellcolor{red!10} 30 & \cellcolor{red!10} 30 & \cellcolor{red!10} 32 & \cellcolor{red!10} 50 & \cellcolor{red!10} 50 & \cellcolor{red!10} 52 & \cellcolor{red!10} 70 & \cellcolor{red!10} 70 & \cellcolor{red!10} 72 & \cellcolor{red!10} 90 & \cellcolor{red!10} 90 & \cellcolor{red!10} 92 \\ \hline
11 & 11 & 13 & \cellcolor{red!30} 31 & 31 & \cellcolor{red!30} 33 & \cellcolor{red!30} 51 & \cellcolor{red!30} 51 & 53 & 71 & 71 & 73 & \cellcolor{red!30} 91 & \cellcolor{red!70} 91 & \cellcolor{red!30} 93 \\ \hline
\cellcolor{red!10} 12 & \cellcolor{red!10} 12 & \cellcolor{red!10} 14 & \cellcolor{red!10} 32 & \cellcolor{red!10} 32 & \cellcolor{red!10} 34 & \cellcolor{red!10} 52 & \cellcolor{red!10} 52 & \cellcolor{red!10} 54 & \cellcolor{red!10} 72 & \cellcolor{red!10} 72 & \cellcolor{red!10} 74 & \cellcolor{red!10} 92 & \cellcolor{red!10} 92 & \cellcolor{red!10} 94 \\ \hline
\cellcolor{red!30} 13 & 13 & \cellcolor{red!30} 15 & \cellcolor{red!30} 33 & \cellcolor{red!30} 33 & \cellcolor{red!50} 35 & \cellcolor{red!50} 53 & 53 & \cellcolor{red!50} 55 & \cellcolor{red!30} 73 & 73 & \cellcolor{red!30} 75 & \cellcolor{red!30} 93 & \cellcolor{red!30} 93 & \cellcolor{red!50} 95 \\ \hline
\cellcolor{red!10} 14 & \cellcolor{red!10} 14 & \cellcolor{red!10} 16 & \cellcolor{red!10} 34 & \cellcolor{red!10} 34 & \cellcolor{red!10} 36 & \cellcolor{red!10} 54 & \cellcolor{red!10} 54 & \cellcolor{red!10} 56 & \cellcolor{red!10} 74 & \cellcolor{red!10} 74 & \cellcolor{red!10} 76 & \cellcolor{red!10} 94 & \cellcolor{red!10} 94 & \cellcolor{red!10} 96 \\ \hline
\cellcolor{red!30} 15 & \cellcolor{red!30} 15 & 17 & \cellcolor{red!50} 35 & \cellcolor{red!50} 35 & 37 & \cellcolor{red!30} 55 & \cellcolor{red!50} 55 & \cellcolor{red!30} 57 & \cellcolor{red!30} 75 & \cellcolor{red!30} 75 & 77 & \cellcolor{red!50} 95 & \cellcolor{red!50} 95 & 97 \\ \hline
\cellcolor{red!10} 16 & \cellcolor{red!10} 16 & \cellcolor{red!10} 18 & \cellcolor{red!10} 36 & \cellcolor{red!10} 36 & \cellcolor{red!10} 38 & \cellcolor{red!10} 56 & \cellcolor{red!10} 56 & \cellcolor{red!10} 58 & \cellcolor{red!10} 76 & \cellcolor{red!10} 76 & \cellcolor{red!10} 78 & \cellcolor{red!10} 96 & \cellcolor{red!10} 96 & \cellcolor{red!10} 98 \\ \hline
17 & 17 & 19 & \cellcolor{red!30} 37 & 37 & \cellcolor{red!30} 39 & \cellcolor{red!30} 57 & \cellcolor{red!30} 57 & 59 & \cellcolor{red!70} 77 & \cellcolor{red!70} 77 & 79 & \cellcolor{red!30} 97 & 97 & \cellcolor{red!30} 99 \\ \hline
\cellcolor{red!10} 18 & \cellcolor{red!10} 18 & \cellcolor{red!10} 20 & \cellcolor{red!10} 38 & \cellcolor{red!10} 38 & \cellcolor{red!10} 40 & \cellcolor{red!10} 58 & \cellcolor{red!10} 58 & \cellcolor{red!10} 60 & \cellcolor{red!10} 78 & \cellcolor{red!10} 78 & \cellcolor{red!10} 80 & \cellcolor{red!10} 98 & \cellcolor{red!10} 98 & \cellcolor{red!10} 100 \\ \hline
\cellcolor{red!30} 19 & 19 & \cellcolor{red!30} 21 & \cellcolor{red!30} 39 & \cellcolor{red!30} 39 & 41 & 59 & 59 & 61 & \cellcolor{red!30} 79 & 79 & \cellcolor{red!30} 81 & \cellcolor{red!30} 99 & \cellcolor{red!30} 99 &   \\ \hline
\cellcolor{red!10} 20 & \cellcolor{red!10} 20 & \cellcolor{red!10} 22 & \cellcolor{red!10} 40 & \cellcolor{red!10} 40 & \cellcolor{red!10} 42 & \cellcolor{red!10} 60 & \cellcolor{red!10} 60 & \cellcolor{red!10} 62 & \cellcolor{red!10} 80 & \cellcolor{red!10} 80 & \cellcolor{red!10} 82 & \cellcolor{red!10} 100 & \cellcolor{red!10} 100 &   \\ \hline
\end{tabular}
\end{adjustbox}
\caption{}
\end{table}

colour coding:

red!10 = \( S_{2} \) and \( S_{2}(2) \)

\vspace{1\baselineskip}

red!30 = \( S_{3} \) and \( S_{3}(2) \)

\vspace{1\baselineskip}

red!50 = \( S_{5} \) and \( S_{5}(2) \)

\vspace{1\baselineskip}

red!70 = \( S_{7} \) and \( S_{7}(2) \) 


\vspace{2\baselineskip}


Given the set 

\begin{equation}
P[S] = P(100) = \{ 2,3,5,7,11,13,17,19,23,29,31,37,41,43,47,53,59,61,67,71,73,79,83,89,97\} ,
\end{equation}

and the shifted sieves:
\begin{align*}
S_{2}(2) & = \{ 2\}, \\
S_{3}(2) & = \{ 7,13,19,31,37,43,61,67,73,79, 97\}, \\
S_{5}(2) & = \{ 23, 53,83\}, \\
S_{7}(2) & = \{ 47,89\},
\end{align*}

we can then derive:

\begin{equation}
P_{1}P[S] = P[S] - \{ S_{2}(2) \cup S_{3}(2) \cup S_{5}(2) \cup S_{7}(2)\} = \{ 3, 5, 11, 17, 29, 41, 59, 71\}.
\end{equation}
\vspace{1\baselineskip}

See the following Table 2.

\begin{table}[H]
\centering
\begin{tabular}{|c|c|c|c|c|c|c|c|c|c|c|c|c|c|}
\hline
p & p+2 &   & p & p+2 &   & p & p+2 &   & p & p+2 &   & p & p+2 \\ \hline
\cellcolor{red!10} 2 & \cellcolor{red!10} 4 &  & \cellcolor{red!30} 13 & \cellcolor{red!30} 15 &  & \cellcolor{red!30} 31 & \cellcolor{red!30} 33 &  & \cellcolor{red!50} 53 & \cellcolor{red!50} 55 &  & \cellcolor{red!30} 73 & \cellcolor{red!30} 75 \\ \hline
3 & 5 &  & 17 & 19 &  & \cellcolor{red!30} 37 & \cellcolor{red!30} 39 &  & 59 & 61 &  & \cellcolor{red!30} 79 & \cellcolor{red!30} 81 \\ \hline
5 & 7 &  & \cellcolor{red!30} 19 & \cellcolor{red!30} 21 &  & 41 & 43 &  & \cellcolor{red!30} 61 & \cellcolor{red!30} 63 &  & \cellcolor{red!50} 83 & \cellcolor{red!50} 85 \\ \hline
\cellcolor{red!30} 7 & \cellcolor{red!30} 9 &  & \cellcolor{red!50} 23 & \cellcolor{red!50} 25 &  & \cellcolor{red!30} 43 & \cellcolor{red!30} 45 &  & \cellcolor{red!30} 67 & \cellcolor{red!30} 69 &  & \cellcolor{red!70} 89 & \cellcolor{red!70} 91 \\ \hline
11 & 13 &   & 29 & 31 &   & \cellcolor{red!70} 47 & \cellcolor{red!70} 49 &   & 71 & 73 &   & \cellcolor{red!30} 97 & \cellcolor{red!30} 99 \\ \hline
\end{tabular}
\caption{}
\end{table}



\vspace{1\baselineskip}

On the other hand, we can find the second prime in the pair by using the Sieve to the following set: 

\begin{equation}
P[S](2) = \left\{ p+2 \, | \, p \in P[S] \right\},
\end{equation}

\begin{equation}
\text{as shown in this set:} \quad Q_{1}P[S] = P[S](2) - \bigcup_{p \in P(\sqrt{n+2})} S_{p},
\end{equation}

\begin{equation}
\text{where} \quad S_{p} = \left\{ mp \, | \, m \in \mathbb{N} \text{ and } p \leq m \leq \frac{n+2}{p} \right\} \text{ for all } p \in P(\sqrt{n+2}).
\end{equation}

\vspace{1\baselineskip}

Take the above Example again:  \( n = 100\),  \( S = \{ 2, 3, \ldots , 99, 100\} ,\)
\vspace{1\baselineskip}

from\ \ \ \ \( P\left[S\right]\left(2\right) = \{ 4,5,7,9,13,15,19,21,25,31,33,39,43,45,49,55,61,63,69,73,75,81,85,91,99\} ,\)
\vspace{1\baselineskip}

and\ \ \ \  \( S_{2} =\left\{ 4\right\} ,  S_{3} =\left\{ 9,15,21,33,39,45,63,69,75,81, 99\right\} ,  S_{5} =\left\{ 25,45,55,75,85\right\} ,  S_{7} =\left\{ 49,63,91\right\} ,\)
\vspace{1\baselineskip}

we have\ \ \ \ \ \ \ \ \(  Q_{1}P\left[S\right] = P\left[S\right](2)-\left\{ S_{2}\cup  S_{3}\cup  S_{5}\cup  S_{7}\right\}  =\left\{ 5, 7, 13, 19, 31, 43, 61, 73\right\} .\)

\vspace{2\baselineskip}

See the following Table 3.


\begin{table}[H]
\centering
\begin{tabular}{|c|c|c|c|c|c|c|c|c|c|c|c|c|c|}
\hline
p & p+2 &   & p & p+2 &   & p & p+2 &   & p & p+2 &   & p & p+2 \\ \hline
2 & \cellcolor{red!10} 4 &  & 13 & \cellcolor{red!30} 15 &  & 31 & \cellcolor{red!30} 33 &  & 53 & \cellcolor{red!50} 55 &  & 73 & \cellcolor{red!30} 75 \\ \hline
3 & 5 &  & 17 & 19 &  & 37 & \cellcolor{red!30} 39 &  & 59 & 61 &  & 79 & \cellcolor{red!30} 81 \\ \hline
5 & 7 &  & 19 & \cellcolor{red!30} 21 &  & 41 & 43 &  & 61 & \cellcolor{red!30} 63 &  & 83 & \cellcolor{red!50} 85 \\ \hline
7 & \cellcolor{red!30} 9 &  & 23 & \cellcolor{red!50} 25 &  & 43 & \cellcolor{red!30} 45 &  & 67 & \cellcolor{red!30} 69 &  & 89 & \cellcolor{red!70} 91 \\ \hline
11 & 13 &   & 29 & 31 &   & 47 & \cellcolor{red!70} 49 &   & 71 & 73 &   & 97 & \cellcolor{red!30} 99 \\ \hline
\end{tabular}
\caption{}
\end{table}




Let \( N_{1} =  N - \{ 1\}\) be the set of natural numbers greater than 1. Then to find all twin primes we can find the first prime in the pair by using  {the Sieves to} \( N_{1}\) as shown in the following set:

\begin{equation}
\left(3\right)         P_{1}P\left[N\right] = N_{1} -  \bigcup_{p\in P}^{} \{  N_{p}\cup  N_{p}(2)\}  = N_{1}- \bigcup_{p\in P}^{} N_{p}- \bigcup_{p\in P}^{}  N_{p}(2) = P- \bigcup_{p\in P}^{}  N_{p}(2), 
\end{equation}

\begin{equation}
\text{where } N_{p} = \left\{ mp \, | \, m \in \mathbb{N} \text{ and } m \geq p \right\} \text{ for all } p \in P,
\end{equation}

\begin{equation}
\text{and } N_{p}(2) = \left\{ mp-2 \, | \, m \in \mathbb{N} \text{ and } m \geq p \right\} \text{ for all } p \in P,
\end{equation}
\vspace{1\baselineskip}


Or we can find the second prime in the pair by using the Sieve to the following set: 

\begin{equation}
P(2) =\left\{  p+2,  p\in P\right\}  ,
\end{equation}



such that            

\begin{equation}
Q_{1}P[N] = P(2) - \bigcup_{p \in P} N_{p},
\end{equation}

\begin{equation}
\text{where } N_{p} = \left\{ mp \, | \, m \in \mathbb{N} \text{ and } m \geq p \right\} \text{ for all } p \in P,
\end{equation}

\vspace{1\baselineskip}

Similarly, for any \( k\in N\) {, to find the} general prime pairs \( (p,  p+2k)\) such that \( p\leq n\), we only need to find the first prime p in the pair by using  {the Sieves to }\( S = \{ m,  1<m\leq n\}\) as shown in the following set:

\begin{equation}
\left(4\right)                P_{k}P\left[S\right] =  S- \bigcup_{p\in P\left(\sqrt{n}\right)}^{} S_{p} - \bigcup_{p\in P\left(\sqrt{n+2k}\right)}^{}S_{p}\left(2k\right) =  P\left[S\right]- \bigcup_{p\in P\left(\sqrt{n+2k}\right)}^{}S_{p}\left(2k\right),
\end{equation}

\begin{equation}
\text{where } S_{p} = \left\{ mp \, | \, m \in \mathbb{N} \text{ and } p \leq m \leq \frac{n}{p} \right\} \text{ for all } p \in P(\sqrt{n}),
\end{equation}

\begin{equation}
S_{p}(2k) = \left\{ mp-2k \, | \, m \in \mathbb{N} \text{ and } \max\left( p, \frac{1+2k}{p} \right) \leq m \leq \frac{n+2k}{p} \right\} \text{ for all } p \in P(\sqrt{n+2k}),
\end{equation}

\vspace{1\baselineskip}

Again, to find all prime pairs \( (p, p+2k) \) we only need to find the first prime in the pair by using the Sieves to \(\mathbb{N}_1\) as shown in the following set:


\begin{equation}
\left(5\right)         P_{k}P\left[N\right] = N_{1}-\bigcup_{p\in P}^{}\{  N_{p}\cup N_{p}(2k)\}  =  N_{1} - \bigcup_{p\in P}^{} N_{p}- \bigcup_{p\in P}^{} N_{p}(2k) = P- \bigcup_{p\in P}^{} N_{p}(2k), 
\end{equation}


\begin{equation}
\text{where } N_{p} = \left\{ mp \, | \, m \in \mathbb{N} \text{ and } m \geq p \right\} \text{ for all } p \in P,
\end{equation}


\begin{equation}
\text{and } N_{p}(2k) = \left\{ mp-2k \, | \, m \in \mathbb{N} \text{ and } m \geq \max\left( p, \frac{1+2k}{p} \right) \right\} \text{ for all } p \in P,
\end{equation}

\begin{center}
Or we can find the second prime in the pair by using  the Sieve to the following set:\textbf{\textit{ \\ }}\begin{equation}
P(2k) =\left\{  p+2k,  p\in P\right\}  ,
\end{equation}


\begin{equation}
Q_{k}P[N] = P(2k) - \bigcup_{p \in P} N_{p},
\end{equation}

\begin{equation}
\text{where } N_{p} = \left\{ mp \, | \, m \in \mathbb{N} \text{ and } m \geq p \right\} \text{ for all } p \in P.
\end{equation}
\end{center}

Let \( \pi_{2}(k;x) \) denote the number of prime pairs \( (p, p+2k) \) such that \( p \leq x \). For the case \( k = 1 \), let \( \pi_{2}(x) = \pi_{2}(1;x) \) denote the number of twin primes \( (p, p+2) \) such that \( p \leq x \). Let \( P_{2}(x) \) be the set of first primes \( p \) in the twin primes \( (p, p+2) \) such that \( p \leq x \).
\vspace{1\baselineskip}

Similar to the Euler \( \phi \)-function, \( \phi(n) \), introduced in the previous section, we define the \( \phi_{2} \)-function, \( \phi_{2}(n) \), as the number of natural numbers \( m \) less than or equal to \( n \) such that both \( m \) and \( m+2 \) are relatively prime to \( n \).
\vspace{1\baselineskip}

For any large positive number \( x \), let 

\begin{equation}
N = \prod_{p \in P(\sqrt{x})} p,
\end{equation}

and

\begin{equation}
X_{2}(z) = \left\{ m \, | \, 1 \leq m \leq z, \gcd(m,N) = 1 \text{ and } \gcd(m+2,N) = 1 \right\}.
\end{equation}
\vspace{1\baselineskip}

From the definition of \( \phi_{2} \)-function, we know that the size of \( X_{2}(N) \) is:


\begin{equation}
 \phi_{2}\left(N\right) = \frac{1}{2} N\prod_{3\leq p\leq\sqrt{x}}^{}(1-\frac{2}{p} ) .
\end{equation}
\vspace{1\baselineskip}

Notice that all numbers in the set \( X_{2}(x) \) are the first primes of twin prime pairs. Moreover, \( X_{2}(x) = P_{2}(x) - P_{2}(\sqrt{x}) \). Thus, the size of \( X_{2}(x) \) is:

\begin{equation}
\pi_{2}(x) - \pi_{2}(\sqrt{x}) = C_{2} \frac{x}{N} \phi_{2}(N) = C_{2} \frac{x}{N} \left\{ \frac{N}{2} \prod_{3 \leq p \leq \sqrt{x}} \left(1-\frac{2}{p}\right) \right\} = \frac{C_{2}x}{2} \prod_{3 \leq p \leq \sqrt{x}} \left(1-\frac{2}{p}\right),
\end{equation}

for some constant \( C_{2} \).
\vspace{1\baselineskip}

Consider the example where \( x = 100 \), \( \sqrt{x} = 10 \), \( P(\sqrt{x}) = \{2,3,5,7\} \), and \( N = 210 \).

\begin{equation}
X_{2}\left(210\right) =\left\{ 11, 17, 29, 41, 59, 71, 101, 107, 137, 149, 167, 179, 191, 197, 209\right\}  = P_{2}\left(210\right)-P_{2}\left(10\right)+\left\{ 167, 209\right\} 
\end{equation}


\begin{equation}
 \phi_{2}\left(210\right) =\frac{210}{2}\left(1-\frac{2}{3}\right)\left(1-\frac{2}{5}\right)\left(1-\frac{2}{7}\right) = 15 ,  
\end{equation}


\begin{equation}
X_{2}\left(100\right) =\left\{ 11, 17, 29, 41, 59, 71\right\}  = P_{2}\left(100\right)- P_{2}\left(10\right),
\end{equation}


the { size of }\( X_{2}\left(100\right)\) is {:} \( \pi_{2}\left(100\right)-\pi_{2}\left(10\right) = 8-2 = 6,\) compare this with

\begin{equation}
C_{2}\frac{100}{210}  \phi_{2}\left(210\right) = C_{2}\frac{100}{210} 15\approx 7.14C_{2} ,  
\end{equation}


we also expect that \(  0<C_{2}<1\).

\vspace{1\baselineskip}

So, like before, if we use the Sieve to S to get \(  P_{1}P\left[S\right]\) as (2), then we have the following approximations of \( \pi_{2}\left(n\right)\), \( \)the size of \(  P_{1}P\left[S\right]\):

\begin{equation}
\frac{C_{2} }{2}n\prod_{3\leq p\leq\sqrt{n}}^{}\left(1-\frac{2}{p}\right) = C_{2} n\left(1-\frac{1}{2}\right)\left(1-\frac{2}{3}\right)\left(1-\frac{2}{5}\right)\ldots \left(1-\frac{2}{q}\right) ,    2<3<\ldots <q\leq\sqrt{n} .  
\end{equation}


Based on the established lemma, we can prove the \textbf{\textit{Twin Prime Conjecture}}.
\vspace{1\baselineskip}

THEOREM  2.1.\ \ \textit{There are infinitely many twin primes.}

\textit{Proof.}\ \ Suppose that there is a finite list of twin primes, and \( q\) is the largest first prime of the last pair.

\begin{center}
Let\(    x = q^{2}\) ,  and \textit{ { \\ }}\begin{equation}
N = \prod_{p\leq q}^{}p
\end{equation}

\end{center}


\begin{equation}
X_{2}\left(z\right) =\left\{  m,  1\leq m\leq z,  \gcd \gcd \left(m,N\right)  = 1  and \gcd \gcd \left(m+2,N\right)  = 1\right\} 
\end{equation}

\vspace{1\baselineskip}

Then all elements in \( X_{2}\left(x\right)\) are the first primes of twin primes,  \( X_{2}\left(x\right) = P_{2}\left(x\right)- P_{2}\left(\sqrt{x}\right)\)\textit{ {,  } {and }}are greater than \( q\); a contradiction. \ \ 

\vspace{1\baselineskip}

\textcolor{red}{Graphics Type `SHAPE' is not supported yet. Please insert it as image.}\par


By using the similar argument of Theorem 1.1, we have the following general result:
\vspace{1\baselineskip}

THEOREM  2.2.\ \ \textit{For any natural number} \textit{k, there are infinitely many pairs of primes that differ by 2k.}

Notice that the above Theorem 1.2 can also be interpreted as follows: 
\vspace{1\baselineskip}

THEOREM  2.3.\ \ \textit{Every even number is the difference of two primes and there are an infinite number of such pairs of primes.\textcolor[HTML]{FFFFFF}{ }}

On the other hand, we want to know if every even number greater than 2 is the sum of two primes, which is also known as the \textit{Goldbach Conjecture} and will be explored in future studies. 

\vspace{1\baselineskip}

 \ \ \ \ Now in order to use the following ([1]):

\begin{equation}
 (6)                                    \prod_{p\leq\sqrt{x}}^{}(1-\frac{1}{p} ) \sim  \frac{2e^{-\gamma }}{\ln \ln  x } ,
\end{equation}


 we need to do some calculations from \( \prod_{}^{}(1-\frac{2}{p} )\) { to }\( \prod_{}^{}(1-\frac{1}{p} )\):

\begin{equation}
\frac{1}{2} \prod_{3\leq p\leq\sqrt{x}}^{}(1-\frac{2}{p} ) = \frac{1\cdot 1\cdot 3\cdots (q-2)}{2\cdot 3\cdot 5\cdots q} = \frac{1\cdot 2\cdot 4\cdots (q-1)}{2\cdot 3\cdot 5\cdots q}\frac{1\cdot 1\cdot 3\cdots (q-2)}{1\cdot 2\cdot 4\cdots (q-1)} 
\end{equation}


\vspace{1\baselineskip}
\begin{equation}
  =  \prod_{p\leq\sqrt{x}}^{}(1-\frac{1}{p} )  \frac{1\cdot 1\cdot 3\cdots (q-2)}{1\cdot 2\cdot 4\cdots (q-1)}   \frac{2\cdot 3\cdot 5\cdots q}{1\cdot 2\cdot 4\cdots (q-1)} \frac{1\cdot 2\cdot 4\cdots (q-1)}{2\cdot 3\cdot 5\cdots q} 
\end{equation}


\begin{equation}
 = 2\left(\prod_{3\leq p\leq\sqrt{x}}^{}\frac{p(p-2)}{(p-1)^{2}}\right)\left(\prod_{p\leq\sqrt{x}}^{}(1-\frac{1}{p} )\right)^{2} .
\end{equation}

\vspace{1\baselineskip}

 Thus, by (6) we have

\begin{equation}
\pi_{2}\left(x\right)-\pi_{2}\left(\sqrt{x}\right)  = \frac{C_{2}x}{2} \prod_{3\leq p\leq\sqrt{x}}^{}\left(1-\frac{2}{p}\right) \sim \frac{8C_{2}}{e^{2\gamma }}\left(\prod_{p\geq 3}^{}\frac{p\left(p-2\right)}{\left(p-1\right)^{2}}\right)\frac{x}{(\ln \ln  x)^{2}} .
\end{equation}


\begin{center}
If let \textit{ \\ }\begin{equation}
A = \frac{8C_{2}}{e^{2\gamma }}\left(\prod_{p\geq 3}^{}\frac{p\left(p-2\right)}{\left(p-1\right)^{2}}\right)  ,
\end{equation}

\end{center}


then

\begin{equation}
                                                 \pi_{2}\left(x\right) \sim  A\frac{x}{(\ln \ln  x)^{2}} .                                                
\end{equation}


\ \ \ \ \ \ \ \ or

\begin{equation}
\left(7\right)                             \pi_{2}\left(x\right) \sim  A \int_{2}^{x}\frac{dt}{(\ln \ln  t)^{2}}  = A\left(Li\left(x\right)-\frac{x}{\ln \ln  x }+\frac{2}{\ln \ln  2 }-Li\left(2\right)\right) ,        
\end{equation}


since 

\begin{equation}
 \int_{}^{}\frac{dt}{(\ln \ln  t)^{2}} =  Li\left(t\right)-\frac{t}{\ln \ln  t }+C  .
\end{equation}

\vspace{1\baselineskip}

 From above (7) we can see that the count of twin primes is determined by the difference of two different approximations of the count of primes,
 
\vspace{1\baselineskip}

\( Li\left(x\right)  and \frac{x}{\ln \ln  x }\) ,

\vspace{1\baselineskip}

For \( \pi_{2}\left(k;n\right)\)\textbf{, }if k has factors 2< p $\leq$\( \sqrt{n}\),  or \(\sqrt{n}<\)p $\leq$\( \sqrt{n+2k} \) then only one p-sequence sieved out from S.  Therefore, we have 

\begin{enumerate}
	\item When k has no other prime factor less than \(\sqrt{n}\) except 2, then

\begin{equation}
\pi_{2}\left(k;n\right) \sim  A_{k} \pi_{2}\left(n\right)
\end{equation}
\vspace{1\baselineskip}

	\item When k has prime factors 2 < p, $\ldots$, q $\leq$\( \sqrt{n}\), then

\end{enumerate}
\begin{equation}
\pi_{2}\left(k;n\right) \sim \frac{\left(p-1\right)\ldots (q-1)  }{\left(p-2\right)\ldots (q-2)  } A_{k} \pi_{2}\left(n\right)
\end{equation}

\begin{flushleft}
where \( A_{k}\) is some constant corresponds to k.
\vspace{1\baselineskip}

Regarding the constant A, we have the following Table 4. showing some examples of the choice of A with A$=$1.32038.  But whether this is the best choice is still unknown. 
\end{flushleft}


\begin{table}[H]
\begin{adjustbox}{max width=\textwidth}
\begin{tabular}{|p{2.53cm}|p{1.9cm}|p{1.9cm}|p{1.67cm}|p{1.82cm}|p{1.48cm}|p{1.67cm}|p{1.93cm}|p{1.75cm}|}
\hline
  & 
\textcolor[HTML]{00B0F0}{ } & 
y$=$ n/ln(n) & 
\textcolor[HTML]{808080}{ } & 
A$=$1.32038 & 
& 
\textcolor[HTML]{808080}{ } & 
A$=$1.32038 & 
\textcolor[HTML]{8DB4E2}{ } \\ 
\hline
n & 
\textcolor[HTML]{00B0F0}{ $\pi$\textsubscript{2}(n)} & 
Li(n)-y & 
\( a\) & 
b$=$A\( a\) & 
\textcolor[HTML]{943734}{b-$\pi$\textsubscript{2}(n)} & 
\textcolor[HTML]{808080}{c} & 
\textcolor[HTML]{808080}{d$=$Ac} & 
\textcolor[HTML]{808080}{d-$\pi$\textsubscript{2}(n)} \\ 
\hline
1,000 & 
\textcolor[HTML]{00B0F0}{35} & 
31 & 
35 & 
46 & 
\textcolor[HTML]{943734}{11} & 
\textcolor[HTML]{808080}{21} & 
\textcolor[HTML]{808080}{28} & 
\textcolor[HTML]{808080}{-7} \\ 
\hline
5,000 & 
\textcolor[HTML]{00B0F0}{126} & 
96 & 
99 & 
131 & 
\textcolor[HTML]{943734}{5} & 
\textcolor[HTML]{808080}{69} & 
\textcolor[HTML]{808080}{91} & 
\textcolor[HTML]{808080}{-35} \\ 
\hline
10,000 & 
\textcolor[HTML]{00B0F0}{205} & 
159 & 
162 & 
214 & 
\textcolor[HTML]{943734}{9} & 
\textcolor[HTML]{808080}{118} & 
\textcolor[HTML]{808080}{156} & 
\textcolor[HTML]{808080}{-49} \\ 
\hline
50,000 & 
\textcolor[HTML]{00B0F0}{705} & 
544 & 
547 & 
722 & 
\textcolor[HTML]{943734}{17} & 
\textcolor[HTML]{808080}{427} & 
\textcolor[HTML]{808080}{564} & 
\textcolor[HTML]{808080}{-141} \\ 
\hline
100,000 & 
\textcolor[HTML]{00B0F0}{1,224} & 
942 & 
946 & 
1249 & 
\textcolor[HTML]{943734}{25} & 
\textcolor[HTML]{808080}{754} & 
\textcolor[HTML]{808080}{996} & 
\textcolor[HTML]{808080}{-228} \\ 
\hline
500,000 & 
\textcolor[HTML]{00B0F0}{4,565} & 
3502 & 
3505 & 
4628 & 
\textcolor[HTML]{943734}{63} & 
\textcolor[HTML]{808080}{2904} & 
\textcolor[HTML]{808080}{3834} & 
\textcolor[HTML]{808080}{-731} \\ 
\hline
1,000,000 & 
\textcolor[HTML]{00B0F0}{8,169} & 
6244 & 
6247 & 
8248 & 
\textcolor[HTML]{943734}{79} & 
\textcolor[HTML]{808080}{5239} & 
\textcolor[HTML]{808080}{6918} & 
\textcolor[HTML]{808080}{-1251} \\ 
\hline
5,000,000 & 
\textcolor[HTML]{00B0F0}{32,463} & 
24487 & 
24490 & 
32336 & 
\textcolor[HTML]{943734}{-127} & 
\textcolor[HTML]{808080}{21015} & 
\textcolor[HTML]{808080}{27747} & 
\textcolor[HTML]{808080}{-4716} \\ 
\hline
10,000,000 & 
\textcolor[HTML]{00B0F0}{58,980} & 
44496 & 
44500 & 
58756 & 
\textcolor[HTML]{943734}{-224} & 
\textcolor[HTML]{808080}{38492} & 
\textcolor[HTML]{808080}{50824} & 
\textcolor[HTML]{808080}{-8156} \\ 
\hline
100,000,000 & 
\textcolor[HTML]{00B0F0}{440,312} & 
333529 & 
333532 & 
440389 & 
\textcolor[HTML]{943734}{77} & 
\textcolor[HTML]{808080}{294706} & 
\textcolor[HTML]{808080}{389124} & 
\textcolor[HTML]{808080}{-51188} \\ 
\hline
10,000,000,000 & 
\textcolor[HTML]{00B0F0}{27,412,679} & 
20761132 & 
20761136 & 
27412589 & 
\textcolor[HTML]{943734}{-90} & 
\textcolor[HTML]{808080}{18861170} & 
\textcolor[HTML]{808080}{24903911} & 
\textcolor[HTML]{808080}{-2508768} \\ 
\hline
\end{tabular}
\end{adjustbox}
\caption{}
\end{table}


\vspace{1\baselineskip}


\begin{equation}
a =  \int_{2}^{n} \frac{dt}{(\ln \ln t)^{2}}, \quad c = \frac{n}{(\ln \ln n)^{2}}
\end{equation}



Prime pairs can be generalized to \href{https://en.wikipedia.org/wiki/Prime_k-tuple}{prime k-tuples}, \( (p_{1},p_{2},p_{3},\cdots  p_{k})\), which are patterns in the differences between more than two prime numbers.
Like the prime pairs, we can study the infinitude and density of prime k-tuples.


\vspace{2\baselineskip}


\section{Prime Pairs with sum to an Even Number}



Next, we consider prime pairs whose sum equals an even number. Again, we employ the Sieve of Eratosthenes to identify such prime pairs.


For a large number \( n \) in \( \mathbb{N} \), consider all pairs \( (a,b) \) satisfying \( a+b = 2n \). Here, \( a \) ranges from \( 2 \) to \( n \), while \( b \) is taken from the set \( \{2n-2, 2n-3, \ldots, n\} \). Using the Sieve, we focus on the set 
\[ 
T = \{ m \,|\, m \in \mathbb{N} \text{ and } 2 \leq m \leq 2n-2 \}
\]
and identify numbers that are sieved out. Following this, we reverse the second half of the set \( \{ n, n+1, \ldots, 2n-2 \} \). The corresponding marks from sets \( \{ a \,|\, 2 \leq a \leq n \} \) and \( \{ b \,|\, 2n-2 \geq b \geq n \} \) are then transferred to the set \( S \), where 
\[ 
S = \{ s \,|\, 2 \leq s \leq n \}.
\]


\vspace{1\baselineskip}

After sieving out all marked down numbers from S, we have all the prime pairs \( (p,q)\) { }such that \( p+q = 2n\)

\vspace{1\baselineskip}

See the following Table 5. (note all even numbers have been sieved out already) and Table 6 for examples.



\begin{table}[H]
\begin{adjustbox}{max width=\textwidth}
\begin{tabular}{p{1.06cm}p{1.06cm}p{1.06cm}p{1.06cm}p{1.06cm}p{1.06cm}p{1.06cm}p{1.06cm}p{1.06cm}p{1.06cm}p{1.06cm}p{1.06cm}p{1.06cm}p{1.06cm}p{1.06cm}}
\hline
\multicolumn{3}{|p{3.17cm}}{\( 60 = a+b\)} & 
\multicolumn{3}{|p{3.17cm}}{\( 62 = a+b\) } & 
\multicolumn{3}{|p{3.17cm}}{\( 64 = a+b\)} & 
\multicolumn{3}{|p{3.17cm}}{\( 66 = a+b\)} & 
\multicolumn{3}{|p{3.17cm}|}{\( 68 = a+b\)} \\ 
\hhline{~~~~~~~~~~~~~~~}
\multicolumn{1}{|p{1.06cm}}{\centering
\textbf{\textcolor[HTML]{00B0F0}{s}}} & 
\multicolumn{1}{p{1.06cm}}{\centering
\textbf{a}} & 
\multicolumn{1}{p{1.06cm}}{\centering
\textbf{b}} & 
\multicolumn{1}{|p{1.06cm}}{\centering
\textbf{\textcolor[HTML]{00B0F0}{s}}} & 
\multicolumn{1}{p{1.06cm}}{\centering
\textbf{a}} & 
\multicolumn{1}{p{1.06cm}}{\centering
\textbf{b}} & 
\multicolumn{1}{|p{1.06cm}}{\centering
\textbf{\textcolor[HTML]{00B0F0}{s}}} & 
\multicolumn{1}{p{1.06cm}}{\centering
\textbf{a}} & 
\multicolumn{1}{p{1.06cm}}{\centering
\textbf{b}} & 
\multicolumn{1}{|p{1.06cm}}{\centering
\textbf{\textcolor[HTML]{00B0F0}{s}}} & 
\multicolumn{1}{p{1.06cm}}{\centering
\textbf{a}} & 
\multicolumn{1}{p{1.06cm}}{\centering
\textbf{b}} & 
\multicolumn{1}{|p{1.06cm}}{\centering
\textbf{\textcolor[HTML]{00B0F0}{s}}} & 
\multicolumn{1}{p{1.06cm}}{\centering
\textbf{a}} & 
\multicolumn{1}{p{1.06cm}|}{\centering
\textbf{b}} \\ 
\hhline{~~~~~~~~~~~~~~~}
\multicolumn{1}{|p{1.06cm}}{\centering
\textcolor[HTML]{00B0F0}{3}} & 
\multicolumn{1}{p{1.06cm}}{\centering
3} & 
\multicolumn{1}{p{1.06cm}}{\centering
57} & 
\multicolumn{1}{|p{1.06cm}}{\centering
\textcolor[HTML]{00B0F0}{3}} & 
\multicolumn{1}{p{1.06cm}}{\centering
3} & 
\multicolumn{1}{p{1.06cm}}{\centering
59} & 
\multicolumn{1}{|p{1.06cm}}{\centering
\textcolor[HTML]{00B0F0}{3}} & 
\multicolumn{1}{p{1.06cm}}{\centering
3} & 
\multicolumn{1}{p{1.06cm}}{\centering
61} & 
\multicolumn{1}{|p{1.06cm}}{\centering
\textcolor[HTML]{00B0F0}{3}} & 
\multicolumn{1}{p{1.06cm}}{\centering
3} & 
\multicolumn{1}{p{1.06cm}}{\centering
63} & 
\multicolumn{1}{|p{1.06cm}}{\centering
\textcolor[HTML]{00B0F0}{3}} & 
\multicolumn{1}{p{1.06cm}}{\centering
3} & 
\multicolumn{1}{p{1.06cm}|}{\centering
65} \\ 
\hhline{~~~~~~~~~~~~~~~}
\multicolumn{1}{|p{1.06cm}}{\centering
\textcolor[HTML]{00B0F0}{5}} & 
\multicolumn{1}{p{1.06cm}}{\centering
5} & 
\multicolumn{1}{p{1.06cm}}{\centering
55} & 
\multicolumn{1}{|p{1.06cm}}{\centering
\textcolor[HTML]{00B0F0}{5}} & 
\multicolumn{1}{p{1.06cm}}{\centering
5} & 
\multicolumn{1}{p{1.06cm}}{\centering
57} & 
\multicolumn{1}{|p{1.06cm}}{\centering
\textcolor[HTML]{00B0F0}{5}} & 
\multicolumn{1}{p{1.06cm}}{\centering
5} & 
\multicolumn{1}{p{1.06cm}}{\centering
59} & 
\multicolumn{1}{|p{1.06cm}}{\centering
\textcolor[HTML]{00B0F0}{5}} & 
\multicolumn{1}{p{1.06cm}}{\centering
5} & 
\multicolumn{1}{p{1.06cm}}{\centering
61} & 
\multicolumn{1}{|p{1.06cm}}{\centering
\textcolor[HTML]{00B0F0}{5}} & 
\multicolumn{1}{p{1.06cm}}{\centering
5} & 
\multicolumn{1}{p{1.06cm}|}{\centering
63} \\ 
\hhline{~~~~~~~~~~~~~~~}
\multicolumn{1}{|p{1.06cm}}{\centering
\textcolor[HTML]{00B0F0}{7}} & 
\multicolumn{1}{p{1.06cm}}{\centering
7} & 
\multicolumn{1}{p{1.06cm}}{\centering
53} & 
\multicolumn{1}{|p{1.06cm}}{\centering
\textcolor[HTML]{00B0F0}{7}} & 
\multicolumn{1}{p{1.06cm}}{\centering
7} & 
\multicolumn{1}{p{1.06cm}}{\centering
55} & 
\multicolumn{1}{|p{1.06cm}}{\centering
\textcolor[HTML]{00B0F0}{7}} & 
\multicolumn{1}{p{1.06cm}}{\centering
7} & 
\multicolumn{1}{p{1.06cm}}{\centering
57} & 
\multicolumn{1}{|p{1.06cm}}{\centering
\textcolor[HTML]{00B0F0}{7}} & 
\multicolumn{1}{p{1.06cm}}{\centering
7} & 
\multicolumn{1}{p{1.06cm}}{\centering
59} & 
\multicolumn{1}{|p{1.06cm}}{\centering
\textcolor[HTML]{00B0F0}{7}} & 
\multicolumn{1}{p{1.06cm}}{\centering
7} & 
\multicolumn{1}{p{1.06cm}|}{\centering
61} \\ 
\hhline{~~~~~~~~~~~~~~~}
\multicolumn{1}{|p{1.06cm}}{\centering
\textcolor[HTML]{00B0F0}{9}} & 
\multicolumn{1}{p{1.06cm}}{\centering
9} & 
\multicolumn{1}{p{1.06cm}}{\centering
51} & 
\multicolumn{1}{|p{1.06cm}}{\centering
\textcolor[HTML]{00B0F0}{9}} & 
\multicolumn{1}{p{1.06cm}}{\centering
9} & 
\multicolumn{1}{p{1.06cm}}{\centering
53} & 
\multicolumn{1}{|p{1.06cm}}{\centering
\textcolor[HTML]{00B0F0}{9}} & 
\multicolumn{1}{p{1.06cm}}{\centering
9} & 
\multicolumn{1}{p{1.06cm}}{\centering
55} & 
\multicolumn{1}{|p{1.06cm}}{\centering
\textcolor[HTML]{00B0F0}{9}} & 
\multicolumn{1}{p{1.06cm}}{\centering
9} & 
\multicolumn{1}{p{1.06cm}}{\centering
57} & 
\multicolumn{1}{|p{1.06cm}}{\centering
\textcolor[HTML]{00B0F0}{9}} & 
\multicolumn{1}{p{1.06cm}}{\centering
9} & 
\multicolumn{1}{p{1.06cm}|}{\centering
59} \\ 
\hhline{~~~~~~~~~~~~~~~}
\multicolumn{1}{|p{1.06cm}}{\centering
\textcolor[HTML]{00B0F0}{11}} & 
\multicolumn{1}{p{1.06cm}}{\centering
11} & 
\multicolumn{1}{p{1.06cm}}{\centering
49} & 
\multicolumn{1}{|p{1.06cm}}{\centering
\textcolor[HTML]{00B0F0}{11}} & 
\multicolumn{1}{p{1.06cm}}{\centering
11} & 
\multicolumn{1}{p{1.06cm}}{\centering
51} & 
\multicolumn{1}{|p{1.06cm}}{\centering
\textcolor[HTML]{00B0F0}{11}} & 
\multicolumn{1}{p{1.06cm}}{\centering
11} & 
\multicolumn{1}{p{1.06cm}}{\centering
53} & 
\multicolumn{1}{|p{1.06cm}}{\centering
\textcolor[HTML]{00B0F0}{11}} & 
\multicolumn{1}{p{1.06cm}}{\centering
11} & 
\multicolumn{1}{p{1.06cm}}{\centering
55} & 
\multicolumn{1}{|p{1.06cm}}{\centering
\textcolor[HTML]{00B0F0}{11}} & 
\multicolumn{1}{p{1.06cm}}{\centering
11} & 
\multicolumn{1}{p{1.06cm}|}{\centering
57} \\ 
\hhline{~~~~~~~~~~~~~~~}
\multicolumn{1}{|p{1.06cm}}{\centering
\textcolor[HTML]{00B0F0}{13}} & 
\multicolumn{1}{p{1.06cm}}{\centering
13} & 
\multicolumn{1}{p{1.06cm}}{\centering
47} & 
\multicolumn{1}{|p{1.06cm}}{\centering
\textcolor[HTML]{00B0F0}{13}} & 
\multicolumn{1}{p{1.06cm}}{\centering
13} & 
\multicolumn{1}{p{1.06cm}}{\centering
49} & 
\multicolumn{1}{|p{1.06cm}}{\centering
\textcolor[HTML]{00B0F0}{13}} & 
\multicolumn{1}{p{1.06cm}}{\centering
13} & 
\multicolumn{1}{p{1.06cm}}{\centering
51} & 
\multicolumn{1}{|p{1.06cm}}{\centering
\textcolor[HTML]{00B0F0}{13}} & 
\multicolumn{1}{p{1.06cm}}{\centering
13} & 
\multicolumn{1}{p{1.06cm}}{\centering
53} & 
\multicolumn{1}{|p{1.06cm}}{\centering
\textcolor[HTML]{00B0F0}{13}} & 
\multicolumn{1}{p{1.06cm}}{\centering
13} & 
\multicolumn{1}{p{1.06cm}|}{\centering
55} \\ 
\hhline{~~~~~~~~~~~~~~~}
\multicolumn{1}{|p{1.06cm}}{\centering
\textcolor[HTML]{00B0F0}{15}} & 
\multicolumn{1}{p{1.06cm}}{\centering
15} & 
\multicolumn{1}{p{1.06cm}}{\centering
45} & 
\multicolumn{1}{|p{1.06cm}}{\centering
\textcolor[HTML]{00B0F0}{15}} & 
\multicolumn{1}{p{1.06cm}}{\centering
15} & 
\multicolumn{1}{p{1.06cm}}{\centering
47} & 
\multicolumn{1}{|p{1.06cm}}{\centering
\textcolor[HTML]{00B0F0}{15}} & 
\multicolumn{1}{p{1.06cm}}{\centering
15} & 
\multicolumn{1}{p{1.06cm}}{\centering
49} & 
\multicolumn{1}{|p{1.06cm}}{\centering
\textcolor[HTML]{00B0F0}{15}} & 
\multicolumn{1}{p{1.06cm}}{\centering
15} & 
\multicolumn{1}{p{1.06cm}}{\centering
51} & 
\multicolumn{1}{|p{1.06cm}}{\centering
\textcolor[HTML]{00B0F0}{15}} & 
\multicolumn{1}{p{1.06cm}}{\centering
15} & 
\multicolumn{1}{p{1.06cm}|}{\centering
53} \\ 
\hhline{~~~~~~~~~~~~~~~}
\multicolumn{1}{|p{1.06cm}}{\centering
\textcolor[HTML]{00B0F0}{17}} & 
\multicolumn{1}{p{1.06cm}}{\centering
17} & 
\multicolumn{1}{p{1.06cm}}{\centering
43} & 
\multicolumn{1}{|p{1.06cm}}{\centering
\textcolor[HTML]{00B0F0}{17}} & 
\multicolumn{1}{p{1.06cm}}{\centering
17} & 
\multicolumn{1}{p{1.06cm}}{\centering
45} & 
\multicolumn{1}{|p{1.06cm}}{\centering
\textcolor[HTML]{00B0F0}{17}} & 
\multicolumn{1}{p{1.06cm}}{\centering
17} & 
\multicolumn{1}{p{1.06cm}}{\centering
47} & 
\multicolumn{1}{|p{1.06cm}}{\centering
\textcolor[HTML]{00B0F0}{17}} & 
\multicolumn{1}{p{1.06cm}}{\centering
17} & 
\multicolumn{1}{p{1.06cm}}{\centering
49} & 
\multicolumn{1}{|p{1.06cm}}{\centering
\textcolor[HTML]{00B0F0}{17}} & 
\multicolumn{1}{p{1.06cm}}{\centering
17} & 
\multicolumn{1}{p{1.06cm}|}{\centering
51} \\ 
\hhline{~~~~~~~~~~~~~~~}
\multicolumn{1}{|p{1.06cm}}{\centering
\textcolor[HTML]{00B0F0}{19}} & 
\multicolumn{1}{p{1.06cm}}{\centering
19} & 
\multicolumn{1}{p{1.06cm}}{\centering
41} & 
\multicolumn{1}{|p{1.06cm}}{\centering
\textcolor[HTML]{00B0F0}{19}} & 
\multicolumn{1}{p{1.06cm}}{\centering
19} & 
\multicolumn{1}{p{1.06cm}}{\centering
43} & 
\multicolumn{1}{|p{1.06cm}}{\centering
\textcolor[HTML]{00B0F0}{19}} & 
\multicolumn{1}{p{1.06cm}}{\centering
19} & 
\multicolumn{1}{p{1.06cm}}{\centering
45} & 
\multicolumn{1}{|p{1.06cm}}{\centering
\textcolor[HTML]{00B0F0}{19}} & 
\multicolumn{1}{p{1.06cm}}{\centering
19} & 
\multicolumn{1}{p{1.06cm}}{\centering
47} & 
\multicolumn{1}{|p{1.06cm}}{\centering
\textcolor[HTML]{00B0F0}{19}} & 
\multicolumn{1}{p{1.06cm}}{\centering
19} & 
\multicolumn{1}{p{1.06cm}|}{\centering
49} \\ 
\hhline{~~~~~~~~~~~~~~~}
\multicolumn{1}{|p{1.06cm}}{\centering
\textcolor[HTML]{00B0F0}{21}} & 
\multicolumn{1}{p{1.06cm}}{\centering
21} & 
\multicolumn{1}{p{1.06cm}}{\centering
39} & 
\multicolumn{1}{|p{1.06cm}}{\centering
\textcolor[HTML]{00B0F0}{21}} & 
\multicolumn{1}{p{1.06cm}}{\centering
21} & 
\multicolumn{1}{p{1.06cm}}{\centering
41} & 
\multicolumn{1}{|p{1.06cm}}{\centering
\textcolor[HTML]{00B0F0}{21}} & 
\multicolumn{1}{p{1.06cm}}{\centering
21} & 
\multicolumn{1}{p{1.06cm}}{\centering
43} & 
\multicolumn{1}{|p{1.06cm}}{\centering
\textcolor[HTML]{00B0F0}{21}} & 
\multicolumn{1}{p{1.06cm}}{\centering
21} & 
\multicolumn{1}{p{1.06cm}}{\centering
45} & 
\multicolumn{1}{|p{1.06cm}}{\centering
\textcolor[HTML]{00B0F0}{21}} & 
\multicolumn{1}{p{1.06cm}}{\centering
21} & 
\multicolumn{1}{p{1.06cm}|}{\centering
47} \\ 
\hhline{~~~~~~~~~~~~~~~}
\multicolumn{1}{|p{1.06cm}}{\centering
\textcolor[HTML]{00B0F0}{23}} & 
\multicolumn{1}{p{1.06cm}}{\centering
23} & 
\multicolumn{1}{p{1.06cm}}{\centering
37} & 
\multicolumn{1}{|p{1.06cm}}{\centering
\textcolor[HTML]{00B0F0}{23}} & 
\multicolumn{1}{p{1.06cm}}{\centering
23} & 
\multicolumn{1}{p{1.06cm}}{\centering
39} & 
\multicolumn{1}{|p{1.06cm}}{\centering
\textcolor[HTML]{00B0F0}{23}} & 
\multicolumn{1}{p{1.06cm}}{\centering
23} & 
\multicolumn{1}{p{1.06cm}}{\centering
41} & 
\multicolumn{1}{|p{1.06cm}}{\centering
\textcolor[HTML]{00B0F0}{23}} & 
\multicolumn{1}{p{1.06cm}}{\centering
23} & 
\multicolumn{1}{p{1.06cm}}{\centering
43} & 
\multicolumn{1}{|p{1.06cm}}{\centering
\textcolor[HTML]{00B0F0}{23}} & 
\multicolumn{1}{p{1.06cm}}{\centering
23} & 
\multicolumn{1}{p{1.06cm}|}{\centering
45} \\ 
\hhline{~~~~~~~~~~~~~~~}
\multicolumn{1}{|p{1.06cm}}{\centering
\textcolor[HTML]{00B0F0}{25}} & 
\multicolumn{1}{p{1.06cm}}{\centering
25} & 
\multicolumn{1}{p{1.06cm}}{\centering
35} & 
\multicolumn{1}{|p{1.06cm}}{\centering
\textcolor[HTML]{00B0F0}{25}} & 
\multicolumn{1}{p{1.06cm}}{\centering
25} & 
\multicolumn{1}{p{1.06cm}}{\centering
37} & 
\multicolumn{1}{|p{1.06cm}}{\centering
\textcolor[HTML]{00B0F0}{25}} & 
\multicolumn{1}{p{1.06cm}}{\centering
25} & 
\multicolumn{1}{p{1.06cm}}{\centering
39} & 
\multicolumn{1}{|p{1.06cm}}{\centering
\textcolor[HTML]{00B0F0}{25}} & 
\multicolumn{1}{p{1.06cm}}{\centering
25} & 
\multicolumn{1}{p{1.06cm}}{\centering
41} & 
\multicolumn{1}{|p{1.06cm}}{\centering
\textcolor[HTML]{00B0F0}{25}} & 
\multicolumn{1}{p{1.06cm}}{\centering
25} & 
\multicolumn{1}{p{1.06cm}|}{\centering
43} \\ 
\hhline{~~~~~~~~~~~~~~~}
\multicolumn{1}{|p{1.06cm}}{\centering
\textcolor[HTML]{00B0F0}{27}} & 
\multicolumn{1}{p{1.06cm}}{\centering
27} & 
\multicolumn{1}{p{1.06cm}}{\centering
33} & 
\multicolumn{1}{|p{1.06cm}}{\centering
\textcolor[HTML]{00B0F0}{27}} & 
\multicolumn{1}{p{1.06cm}}{\centering
27} & 
\multicolumn{1}{p{1.06cm}}{\centering
35} & 
\multicolumn{1}{|p{1.06cm}}{\centering
\textcolor[HTML]{00B0F0}{27}} & 
\multicolumn{1}{p{1.06cm}}{\centering
27} & 
\multicolumn{1}{p{1.06cm}}{\centering
37} & 
\multicolumn{1}{|p{1.06cm}}{\centering
\textcolor[HTML]{00B0F0}{27}} & 
\multicolumn{1}{p{1.06cm}}{\centering
27} & 
\multicolumn{1}{p{1.06cm}}{\centering
39} & 
\multicolumn{1}{|p{1.06cm}}{\centering
\textcolor[HTML]{00B0F0}{27}} & 
\multicolumn{1}{p{1.06cm}}{\centering
27} & 
\multicolumn{1}{p{1.06cm}|}{\centering
41} \\ 
\hhline{~~~~~~~~~~~~~~~}
\multicolumn{1}{|p{1.06cm}}{\centering
\textcolor[HTML]{00B0F0}{29}} & 
\multicolumn{1}{p{1.06cm}}{\centering
29} & 
\multicolumn{1}{p{1.06cm}}{\centering
31} & 
\multicolumn{1}{|p{1.06cm}}{\centering
\textcolor[HTML]{00B0F0}{29}} & 
\multicolumn{1}{p{1.06cm}}{\centering
29} & 
\multicolumn{1}{p{1.06cm}}{\centering
33} & 
\multicolumn{1}{|p{1.06cm}}{\centering
\textcolor[HTML]{00B0F0}{29}} & 
\multicolumn{1}{p{1.06cm}}{\centering
29} & 
\multicolumn{1}{p{1.06cm}}{\centering
35} & 
\multicolumn{1}{|p{1.06cm}}{\centering
\textcolor[HTML]{00B0F0}{29}} & 
\multicolumn{1}{p{1.06cm}}{\centering
29} & 
\multicolumn{1}{p{1.06cm}}{\centering
37} & 
\multicolumn{1}{|p{1.06cm}}{\centering
\textcolor[HTML]{00B0F0}{29}} & 
\multicolumn{1}{p{1.06cm}}{\centering
29} & 
\multicolumn{1}{p{1.06cm}|}{\centering
39} \\ 
\hhline{~~~~~~~~~~~~~~~}
\multicolumn{1}{|p{1.06cm}}{} & 
\multicolumn{1}{p{1.06cm}}{} & 
\multicolumn{1}{p{1.06cm}}{} & 
\multicolumn{1}{|p{1.06cm}}{\centering
\textcolor[HTML]{00B0F0}{31}} & 
\multicolumn{1}{p{1.06cm}}{\centering
31} & 
\multicolumn{1}{p{1.06cm}}{\centering
31} & 
\multicolumn{1}{|p{1.06cm}}{\centering
\textcolor[HTML]{00B0F0}{31}} & 
\multicolumn{1}{p{1.06cm}}{\centering
31} & 
\multicolumn{1}{p{1.06cm}}{\centering
33} & 
\multicolumn{1}{|p{1.06cm}}{\centering
\textcolor[HTML]{00B0F0}{31}} & 
\multicolumn{1}{p{1.06cm}}{\centering
31} & 
\multicolumn{1}{p{1.06cm}}{\centering
35} & 
\multicolumn{1}{|p{1.06cm}}{\centering
\textcolor[HTML]{00B0F0}{31}} & 
\multicolumn{1}{p{1.06cm}}{\centering
31} & 
\multicolumn{1}{p{1.06cm}|}{\centering
37} \\ 
\hhline{~~~~~~~~~~~~~~~}
\multicolumn{1}{|p{1.06cm}}{} & 
\multicolumn{1}{p{1.06cm}}{} & 
\multicolumn{1}{p{1.06cm}}{} & 
\multicolumn{1}{|p{1.06cm}}{} & 
\multicolumn{1}{p{1.06cm}}{} & 
\multicolumn{1}{p{1.06cm}}{} & 
\multicolumn{1}{|p{1.06cm}}{} & 
\multicolumn{1}{p{1.06cm}}{} & 
\multicolumn{1}{p{1.06cm}}{} & 
\multicolumn{1}{|p{1.06cm}}{\centering
\textcolor[HTML]{00B0F0}{33}} & 
\multicolumn{1}{p{1.06cm}}{\centering
33} & 
\multicolumn{1}{p{1.06cm}}{\centering
33} & 
\multicolumn{1}{|p{1.06cm}}{\centering
\textcolor[HTML]{00B0F0}{33}} & 
\multicolumn{1}{p{1.06cm}}{\centering
33} & 
\multicolumn{1}{p{1.06cm}|}{\centering
35} \\ 
\hline
\end{tabular}
\end{adjustbox}
\caption{}
\end{table}

\vspace{1\baselineskip}


\begin{table}[H]
\begin{adjustbox}{max width=\textwidth}
\begin{tabular}{p{1.06cm}p{1.06cm}p{1.06cm}p{1.06cm}p{1.06cm}p{1.06cm}p{1.06cm}p{1.06cm}p{1.06cm}p{1.06cm}p{1.06cm}p{1.06cm}p{1.06cm}p{1.06cm}p{1.06cm}}
\hline
\multicolumn{15}{|p{15.87cm}|}{\centering
   After the Sieves, we have the following left prime pairs:} \\ 
\hline
\multicolumn{3}{|p{3.17cm}}{\( 60 = p+q\)} & 
\multicolumn{3}{|p{3.17cm}}{\( 62 = p+q\)} & 
\multicolumn{3}{|p{3.17cm}}{\( 64 = p+q\)} & 
\multicolumn{3}{|p{3.17cm}}{\( 66 = p+q\)} & 
\multicolumn{3}{|p{3.17cm}|}{\( 68 = p+q\)} \\ 
\hhline{~~~~~~~~~~~~~~~}
\multicolumn{1}{|p{1.06cm}}{\centering
\textbf{\textit{\textcolor[HTML]{00B0F0}{s}}}} & 
\multicolumn{1}{p{1.06cm}}{\centering
\textbf{\textit{p}}} & 
\multicolumn{1}{p{1.06cm}}{\centering
\textbf{\textit{q}}} & 
\multicolumn{1}{|p{1.06cm}}{\centering
\textbf{\textit{\textcolor[HTML]{00B0F0}{s}}}} & 
\multicolumn{1}{p{1.06cm}}{\centering
\textbf{\textit{p}}} & 
\multicolumn{1}{p{1.06cm}}{\centering
\textbf{\textit{q}}} & 
\multicolumn{1}{|p{1.06cm}}{\centering
\textbf{\textit{\textcolor[HTML]{00B0F0}{s}}}} & 
\multicolumn{1}{p{1.06cm}}{\centering
\textbf{\textit{p}}} & 
\multicolumn{1}{p{1.06cm}}{\centering
\textbf{\textit{q}}} & 
\multicolumn{1}{|p{1.06cm}}{\centering
\textbf{\textit{\textcolor[HTML]{00B0F0}{s}}}} & 
\multicolumn{1}{p{1.06cm}}{\centering
\textbf{\textit{p}}} & 
\multicolumn{1}{p{1.06cm}}{\centering
\textbf{\textit{q}}} & 
\multicolumn{1}{|p{1.06cm}}{\centering
\textbf{\textit{\textcolor[HTML]{00B0F0}{s}}}} & 
\multicolumn{1}{p{1.06cm}}{\centering
\textbf{\textit{p}}} & 
\multicolumn{1}{p{1.06cm}|}{\centering
\textbf{\textit{q}}} \\ 
\hhline{~~~~~~~~~~~~~~~}
\multicolumn{1}{|p{1.06cm}}{\centering
\textcolor[HTML]{00B0F0}{7}} & 
\multicolumn{1}{p{1.06cm}}{\centering
7} & 
\multicolumn{1}{p{1.06cm}}{\centering
53} & 
\multicolumn{1}{|p{1.06cm}}{\centering
\textcolor[HTML]{00B0F0}{3}} & 
\multicolumn{1}{p{1.06cm}}{\centering
3} & 
\multicolumn{1}{p{1.06cm}}{\centering
59} & 
\multicolumn{1}{|p{1.06cm}}{\centering
\textcolor[HTML]{00B0F0}{3}} & 
\multicolumn{1}{p{1.06cm}}{\centering
3} & 
\multicolumn{1}{p{1.06cm}}{\centering
61} & 
\multicolumn{1}{|p{1.06cm}}{\centering
\textcolor[HTML]{00B0F0}{5}} & 
\multicolumn{1}{p{1.06cm}}{\centering
5} & 
\multicolumn{1}{p{1.06cm}}{\centering
61} & 
\multicolumn{1}{|p{1.06cm}}{\centering
\textcolor[HTML]{00B0F0}{7}} & 
\multicolumn{1}{p{1.06cm}}{\centering
7} & 
\multicolumn{1}{p{1.06cm}|}{\centering
61} \\ 
\hhline{~~~~~~~~~~~~~~~}
\multicolumn{1}{|p{1.06cm}}{\centering
\textcolor[HTML]{00B0F0}{13}} & 
\multicolumn{1}{p{1.06cm}}{\centering
13} & 
\multicolumn{1}{p{1.06cm}}{\centering
47} & 
\multicolumn{1}{|p{1.06cm}}{\centering
\textcolor[HTML]{00B0F0}{19}} & 
\multicolumn{1}{p{1.06cm}}{\centering
19} & 
\multicolumn{1}{p{1.06cm}}{\centering
43} & 
\multicolumn{1}{|p{1.06cm}}{\centering
\textcolor[HTML]{00B0F0}{5}} & 
\multicolumn{1}{p{1.06cm}}{\centering
5} & 
\multicolumn{1}{p{1.06cm}}{\centering
59} & 
\multicolumn{1}{|p{1.06cm}}{\centering
\textcolor[HTML]{00B0F0}{7}} & 
\multicolumn{1}{p{1.06cm}}{\centering
7} & 
\multicolumn{1}{p{1.06cm}}{\centering
59} & 
\multicolumn{1}{|p{1.06cm}}{\centering
\textcolor[HTML]{00B0F0}{31}} & 
\multicolumn{1}{p{1.06cm}}{\centering
31} & 
\multicolumn{1}{p{1.06cm}|}{\centering
37} \\ 
\hhline{~~~~~~~~~~~~~~~}
\multicolumn{1}{|p{1.06cm}}{\centering
\textcolor[HTML]{00B0F0}{17}} & 
\multicolumn{1}{p{1.06cm}}{\centering
17} & 
\multicolumn{1}{p{1.06cm}}{\centering
43} & 
\multicolumn{1}{|p{1.06cm}}{\centering
\textcolor[HTML]{00B0F0}{31}} & 
\multicolumn{1}{p{1.06cm}}{\centering
31} & 
\multicolumn{1}{p{1.06cm}}{\centering
31} & 
\multicolumn{1}{|p{1.06cm}}{\centering
\textcolor[HTML]{00B0F0}{11}} & 
\multicolumn{1}{p{1.06cm}}{\centering
11} & 
\multicolumn{1}{p{1.06cm}}{\centering
53} & 
\multicolumn{1}{|p{1.06cm}}{\centering
\textcolor[HTML]{00B0F0}{13}} & 
\multicolumn{1}{p{1.06cm}}{\centering
13} & 
\multicolumn{1}{p{1.06cm}}{\centering
53} & 
\multicolumn{1}{|p{1.06cm}}{} & 
\multicolumn{1}{p{1.06cm}}{} & 
\multicolumn{1}{p{1.06cm}|}{} \\ 
\hhline{~~~~~~~~~~~~~~~}
\multicolumn{1}{|p{1.06cm}}{\centering
\textcolor[HTML]{00B0F0}{19}} & 
\multicolumn{1}{p{1.06cm}}{\centering
19} & 
\multicolumn{1}{p{1.06cm}}{\centering
41} & 
\multicolumn{1}{|p{1.06cm}}{} & 
\multicolumn{1}{p{1.06cm}}{} & 
\multicolumn{1}{p{1.06cm}}{} & 
\multicolumn{1}{|p{1.06cm}}{\centering
\textcolor[HTML]{00B0F0}{17}} & 
\multicolumn{1}{p{1.06cm}}{\centering
17} & 
\multicolumn{1}{p{1.06cm}}{\centering
47} & 
\multicolumn{1}{|p{1.06cm}}{\centering
\textcolor[HTML]{00B0F0}{19}} & 
\multicolumn{1}{p{1.06cm}}{\centering
19} & 
\multicolumn{1}{p{1.06cm}}{\centering
47} & 
\multicolumn{1}{|p{1.06cm}}{} & 
\multicolumn{1}{p{1.06cm}}{} & 
\multicolumn{1}{p{1.06cm}|}{} \\ 
\hhline{~~~~~~~~~~~~~~~}
\multicolumn{1}{|p{1.06cm}}{\centering
\textcolor[HTML]{00B0F0}{23}} & 
\multicolumn{1}{p{1.06cm}}{\centering
23} & 
\multicolumn{1}{p{1.06cm}}{\centering
37} & 
\multicolumn{1}{|p{1.06cm}}{} & 
\multicolumn{1}{p{1.06cm}}{} & 
\multicolumn{1}{p{1.06cm}}{} & 
\multicolumn{1}{|p{1.06cm}}{\centering
\textcolor[HTML]{00B0F0}{23}} & 
\multicolumn{1}{p{1.06cm}}{\centering
23} & 
\multicolumn{1}{p{1.06cm}}{\centering
41} & 
\multicolumn{1}{|p{1.06cm}}{\centering
\textcolor[HTML]{00B0F0}{23}} & 
\multicolumn{1}{p{1.06cm}}{\centering
23} & 
\multicolumn{1}{p{1.06cm}}{\centering
43} & 
\multicolumn{1}{|p{1.06cm}}{} & 
\multicolumn{1}{p{1.06cm}}{} & 
\multicolumn{1}{p{1.06cm}|}{} \\ 
\hhline{~~~~~~~~~~~~~~~}
\multicolumn{1}{|p{1.06cm}}{\centering
\textcolor[HTML]{00B0F0}{29}} & 
\multicolumn{1}{p{1.06cm}}{ 29} & 
\multicolumn{1}{p{1.06cm}}{\centering
31} & 
\multicolumn{1}{|p{1.06cm}}{} & 
\multicolumn{1}{p{1.06cm}}{} & 
\multicolumn{1}{p{1.06cm}}{} & 
\multicolumn{1}{|p{1.06cm}}{} & 
\multicolumn{1}{p{1.06cm}}{} & 
\multicolumn{1}{p{1.06cm}}{} & 
\multicolumn{1}{|p{1.06cm}}{\centering
\textcolor[HTML]{00B0F0}{29}} & 
\multicolumn{1}{p{1.06cm}}{\centering
29} & 
\multicolumn{1}{p{1.06cm}}{\centering
37} & 
\multicolumn{1}{|p{1.06cm}}{} & 
\multicolumn{1}{p{1.06cm}}{} & 
\multicolumn{1}{p{1.06cm}|}{} \\ 
\hline
\end{tabular}
\end{adjustbox}
\caption{}
\end{table}

\vspace{1\baselineskip}


\section{Sums of Primes}


 The above Sieve process also can be expressed in the following way.
 
\vspace{1\baselineskip}

Like before,  {to find }all the prime pairs \( (p,q)\) { }such that \( p+q = 2n\), we only need to find the first prime \textit{p} in the pair by using  {the Sieves to the set }\( S =\left\{ s,  2\leq s\leq n\right\}\) {\ \ }as shown in the following  {set}:


\begin{equation}
PP[S] = S - \bigcup_{p \in P(\sqrt{n})} S_{p} - \bigcup_{p \in P(\sqrt{2n})} S_{p}(x_{p}) = P[S] - \bigcup_{p \in P(\sqrt{2n})} S_{p}(x_{p})
\end{equation}

\begin{equation}
\text{where } S_{p} = \left\{ mp, m \in N \text{ and } p \leq m \leq \frac{n}{p} \right\} \text{ for all } p \in P(\sqrt{n}),
\end{equation}

\begin{equation}
\text{and } S_{p}(x_{p}) = \left\{ mp+x_{p}, m \in N \text{ and } m \leq \frac{n-x_{p}}{p}, \text{ for some } -p < x_{p} < p \right\} \text{ for all } p \in P(\sqrt{2n}).
\end{equation}

\vspace{1\baselineskip}


Notice that \( x_{p}\) varies depending on the n, but we always have \( x_{2 } = 0\).
\vspace{1\baselineskip}

For example, in Table 2:
\vspace{1\baselineskip}
 
When:
\vspace{1\baselineskip}
\begin{center}
    

( n = 30), then \( x_{2 } = x_{3 } = x_{5 } = 0\) and \( x_{7 } = 4\);

( n = 32), then \( x_{2 } = 0,  x_{3 } = x_{7 } = 1\) and \( x_{5 } = 4\);

( n = 33), then \( x_{2 } = x_{3 } = 0,  x_{5 } = 1\) and \( x_{7 } = 3\);

( n = 34), then \( x_{2 } = 0,  x_{3 } = 2,  x_{5 } = -2\) and \( x_{7 } = -2.\)

\end{center}

 \vspace{1\baselineskip}

For \( n = 31 \),

\begin{align}
S &= \{ a \} = \{ 2,3,\ldots,30,31 \}, \\
\{ b \} &= \{ 60,59,\ldots,32,31 \}, \\
T &= \{ a \} \cup \{ b \} = \{ 2,3,\ldots,31,\ldots,60 \},
\end{align}

\vspace{1\baselineskip}

Since \( x_{2} = 0 \), \( x_{3} = x_{5} = 2 \), and \( x_{7} = 6 \), we have

\begin{align}
S_{2} &= \{ 4, 6, \ldots, 30 \}, \\
S_{3} &= \{ 9, 12, \ldots, 27, 30 \}, \\
S_{2}(0) &= \{ 2, 4, \ldots, 30 \}, \\
S_{3}(2) &= \{ 5, 8, \ldots, 23, 26, 29 \}, \\
S_{5} &= \{ 25 \}, \\
S_{5}(2) &= \{ 7, 12, 17, 22, 27 \}, \\
S_{7}(6) &= \{ 13, 20, 27 \}.
\end{align}

\vspace{1\baselineskip}

From 
\[
P[S] = S - \left( S_{2} \cup S_{2}(0) \cup S_{3} \cup S_{3}(2) \cup S_{5} \cup S_{5}(2) \cup S_{7}(6) \right) = \{ 3, 19, 31 \},
\]


\vspace{1\baselineskip}

we find out all 3 prime pairs:

\(\left\{\left(3, 59\right),\left(19, 43\right),\left(31, 31\right)\right\}\) \textit{\ \ \ \ }
\vspace{1\baselineskip}

such that \ \ \( 62 = 3+59 = 19+43 = 31+31\).

\vspace{1\baselineskip}

For \( n\geq 2\), let \( \eta (2n)\) denote the number of prime pairs with the sum equal to \( 2n\),  and we can call it the \textbf{\textit{Goldbach number}} since it comes from the Goldbach conjecture. 

\vspace{1\baselineskip}

For example,\ \ \ \ \ \ \( \eta\left(4\right) = \eta\left(6\right) = \eta\left(8\right) = 1,  \eta\left(10\right) = 2,  \eta\left(12\right) = 1, \ldots \ldots\)
\vspace{1\baselineskip}

From Table 3, we have \( \eta\left(60\right) = 6,  \eta\left(62\right) = 3,  \eta\left(64\right) = 5,  \eta\left(66\right) = 6,  \eta\left(68\right) = 2.\)

\vspace{1\baselineskip}


Since \( \eta (2n)\)\textit{ }is the size of the set PP[S] from (8), then 

\begin{enumerate}
	\item Like \( \pi_{2}\left(k;n\right)\),\textbf{ }\( \eta (2n)\)\textbf{ }also\textbf{ }varies with \( n\) depending on the factors of \( n\).  If \( 2<p\leq\sqrt{n}\) is a factor of \( n\),  or \(\sqrt{n}<p\leq\sqrt{2n} , \)then only one p-sequence sieved out from S.  

\vspace{1\baselineskip}
 For example,\ \( n = 30\) has factors 3 and 5, and \(\sqrt{30}<7\leq\sqrt{60} , \)so only one 3-sequence, 5-sequence, and 7-sequence sieved out, that \( \eta\left(60\right) = 6\) (see Table 5).  
 
\vspace{1\baselineskip}

	\item Also, from Table 5 we can see that when \( n = 32\), some elements from 5-sequence and 7-sequence are merged into 3-sequence so that less elements are sieved out.

\end{enumerate}

\vspace{1\baselineskip}

While other than 1), 2) above, in general for each prime \( 3\leq p\leq\sqrt{n}\) , there are at most two p-sequences are sieved out from S, and therefore, we have the following approximations of \( \eta (2n)\):
\vspace{1\baselineskip}
\begin{equation}
\frac{C }{2} n\left\{  \prod_{3\leq p\leq\sqrt{n}}^{}\left(1-\frac{2}{p}\right)\right\}\left\{ \prod_{\sqrt{n}<q\leq\sqrt{2n}}^{}\left(1-\frac{1}{q}\right)\right\}  \sim  G\frac{n}{(\ln \ln  n)^{2}}   ,
\end{equation}

\vspace{1\baselineskip}

where C and G are some corresponding constants.
\vspace{1\baselineskip}

For 1) when n has prime factors \( 2<p, \ldots , q\leq\sqrt{n}\) {,  then }

\begin{equation}
\eta\left(2n\right)  \sim \frac{\left(p-1\right)\ldots (q-1)  }{\left(p-2\right)\ldots (q-2)  } G\frac{n}{(\ln \ln  n)^{2}} > G\frac{n}{(\ln \ln  n)^{2}} ;
\end{equation}

\vspace{1\baselineskip}


\ For 2) when some elements from p-sequence are merged into other q-sequence, then


\begin{equation}
    \eta\left(2n\right)> G\frac{n}{(\ln \ln  n)^{2}}  .
\end{equation}

\vspace{1\baselineskip}

Therefore, when n is large, we have the following:

\begin{equation}
    \eta\left(2n\right)\geq  G\frac{n}{(\ln \ln  n)^{2}}>1 .
\end{equation}
\vspace{1\baselineskip}

The appendix shows some examples of \( \eta\left(2n\right)\) and \(\frac{n}{(\ln \ln  n)^{2}}\) such that \( \eta\left(2n\right)>\frac{n}{(\ln \ln  n)^{2}}\)\ \ with only a few exceptions.  
\vspace{1\baselineskip}

So, we have the following:
\vspace{1\baselineskip}

THEOREM 3.1.  \textit{For any even number }n > 2, \textit{there exists }\( \eta\left(n\right)\geq 1\)\textit{ of prime pairs p and q such that }

\textit{n $=$ p + q.}

\vspace{1\baselineskip}

This is much stronger than the following \textit{Goldbach’s conjecture}. 

\vspace{1\baselineskip}

THEOREM 3.2.  \textit{Every even number greater than 2 can be written as the sum of two primes.}

\newpage

Acknowledgments. Thank you to the reviewers for carefully reading the manuscript and making useful suggestions.


\bibliography{lal_aom_01}
\bibliographystyle{aomalpha}


\textcolor[HTML]{333333}{[1] G. H. Hardy and E. M. Wright, }\textit{An Introduction to the Theory of Numbers}\textcolor[HTML]{333333}{, 6\textsuperscript{th} ed. Oxford University Press, 2008. }

\vspace{1\baselineskip}

\textcolor[HTML]{333333}{[2] G. E. Andrews, }\textit{Number Theory, }1\textsuperscript{st} Dover ed. Dover Publications, 1994.

\vspace{1\baselineskip}

\textcolor[HTML]{333333}{[3] P. X. Gallagher, }\textit{On the distribution of primes in short intervals,} Mathematika, vol. 23, no. 1, pp. 4–9, 1976.

\vspace{1\baselineskip}

\textcolor[HTML]{333333}{[4] E. Landau, }\textit{Handbuch der Lehre von der Verteilung der Primzahlen,} 2 vols. B. G. Teubner, 1909.

\vspace{1\baselineskip}

\textcolor[HTML]{333333}{[5] J. E. Littlewood, }\textit{On the distribution of prime numbers,} Mathematische Zeitschrift, vol. 1, no. 5, pp. 281–295, 1914.

\vspace{1\baselineskip}

\textcolor[HTML]{333333}{[6] A. M. Odlyzko, }\textit{The 10\textsuperscript{20}th zero of the Riemann zeta function and 175 million of its neighbors,} unpublished.

\vspace{1\baselineskip}

\textcolor[HTML]{333333}{[7] M. Rubinstein and P. Sarnak, }\textit{Chebyshev’s bias,} Experimental Mathematics, vol. 3, no. 3, pp. 173–197, 1994.

\vspace{1\baselineskip}

\textcolor[HTML]{333333}{[8] R. C. Baker, G. Harman, and J. Pintz, }\textit{The difference between consecutive primes, II,} Proceedings of the London Mathematical Society, vol. 83, no. 3, pp. 532–562, 2001.

\vspace{1\baselineskip}

\textcolor[HTML]{333333}{[9] H. Riesel, }\textit{Prime Numbers and Computer Methods for Factorization,} 2nd ed. Birkhäuser, 1994.

\vspace{1\baselineskip}

\textcolor[HTML]{333333}{[10] A. Granville and G. Martin, }\textit{Prime number races,} American Mathematical Monthly, vol. 113, no. 1, pp. 1–33, 2006.




\newpage
\appendix
\centering
\textbf{Appendix}
\input{appendix_01}

\vspace{1\baselineskip}


\end{document}
